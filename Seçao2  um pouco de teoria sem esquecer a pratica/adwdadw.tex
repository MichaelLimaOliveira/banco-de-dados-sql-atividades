Nome do Livro Nome do Autor Sexo do Autor Numero de Páginas Nome da Editora Valor do Livro UF da Editora Ano da Publicacao
1 Cavaleiro Real
2 5QL para leigos
3 Receitas Caseiras Celia Tavares Feminino
4 Pessoas Efetivas
5 Habitos Saudáveis Eduardo Santos Masculino
6 A Casa Marrom
7 Estacio Querido
8 Pra sempre amigas Leda Silva
9 Copas inesqueciveis Marco Alcantara Masculino
10 O poder da mente Clara Mafra

Ana Claudia_
João Nunes
465 Atlas
Feminino
Masculino
49.9 R
98 SP
45 RJ
20
o18
450|Addison
210 Atlas
390|Beta
Eduardo SantosMasculino
630 Beta
250|Bubba
78.9|R
150.98 RJ
60 MG
100 ES
2019
Hermes Macedo Masculino
Geraldo Francisco | Masculino
2016
310|Insignia
510 insignia
201S
2011
2018
2017
Feminino
78.98ES
130.98 RS
200|Larson
120 Continental
56.58 RS
Feminino



Impacto ambiental é um desequilíbrio provocado pela ação dos seres humanos sobre o meio ambiente.
Pode resultar também de acidentes naturais: a explosão de um vulcão pode provocar poluição atmosférica; 
o choque de um meteoro, destruição de animais e vegetais; um raio, um incêndio numa floresta.
    Quando os ecossistemas sofrem impactos ambientais, geralmente a vegetação é o primeiro elemento da
natureza a ser atingido, pois é reflexo combinado das condições naturais de solo, relevo e clima do lugar 
em que ocorre.
    Atualmente todas as formações vegetais, em maior ou menor grau, encontram-se modificadas pela ação humana.
Isso ocorreu principalmente por causa das atividades agropecuárias e pelos impactos causados pela industrialização
e urbanização. Em muitos casos, sobram apenas algumas manchas em que a vegetação original é encontrada, nos quais,
embora com pequenas alterações, ainda preserva suas características principais.
    A primeira consequência do desmatamento é o comprometimento da biodiversidade, por causa da diminuição ou, mesmo,
da extinção de espécies vegetais e animais. As florestas tropicais tem uma enorme biodiversidade e, por isso, possuem
um valor incalculável. Muitas espécies, hoje ainda desconhecidas da sociedade urbano-industrial, podem vir a ser a solução
para a cura de doenças e poderão ser usadas na alimentação ou como matérias primas. Com o desmatamento, há o risco dessas
espécies serem destruídas antes de serem descobertas e estudadas.
    Particularmente na Floresta Amazônica há uma enorme quantidade de espécies endêmicas. Parte desse patrimônio genético
é conhecido pelas várias nações indígenas que ali habitavam, mas a maioria dessas comunidades nativas está sofrendo um
processo de integração à sociedade urbano-industrial que tem levado a perda do patrimônio dos seus conhecimentos. Outro
ponto importante que afeta os interesses nacionais dos países onde há florestas tropicais, incluindo o Brasil, é a 
biopirataria, por meio da qual muitas empresas adotam práticas ilegais para garantir o direito de explorar, futuramente,
uma possível matéria prima para a indústria farmacêutica e de cosméticos, entre outras.
    No Brasil, os incêndios ou queimadas de florestas, que consomem uma quantidade incalculável de biomassa todos os
anos, são provocados para o desenvolvimento de atividades agropecuárias, muitas vezes em grandes projetos que recebem
incentivos governamentais e, portanto, sob o amparo da lei. Podem também ser resultado de práticas criminosas ou ainda
de acidentes, inclusive naturais.
    A devastação já ocorrida deve-se basicamente a fatores econômicos tanto na Amazônia quanto nas florestas africanas
e nas do Sul e Sudeste Asiático. Suas principais causas são:
- Extração de madeira;
- Instalação de projetos agropecuários;
- Implantação de projetos de mineração;
- Instalação ou expansão de garimpos;
- Construção de usinas hidrelétricas;
- Incêndios;
- Queimadas (técnica de cultivo tradicional).
As consequências socioambientais das interferências humanas em regiões de florestas são várias:
- Aumento do processo erosivo, o que leva a um empobrecimento do solo;
- Assoreamento dos rios e lagos, que resulta do aumento da sedimentação, que provoca enchentes;
- Rebaixamento do aquífero, causada por menor infiltração de água das chuvas no subsolo;
- Diminuição dos índices pluviométricos, em consequência do fim da transpiração das plantas;
- Elevação das temperaturas locais e regionais, como consequência da maior irradiação de calor para a atmosfera a 
partir do solo exposto.
- Agravamento do processo de desertificação;